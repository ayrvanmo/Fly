\section{Introducción}
\rhostart{}

\textbf{Fly} es un motor de búsqueda que indexa documentos web con hipervínculos a otros documentos, utilizando técnicas como el \textbf{índice invertido} y \textbf{PageRank}. Un sistema de recuperación de información es una herramienta que permite buscar y recuperar datos en una colección de archivos. Esto se utiliza en aplicaciones como buscadores web y bases de datos.

Las principales funciones de \textbf{Fly} incluyen:
\begin{itemize}
\item Procesar documentos de un directorio y extraer palabras clave.
\item Crear un índice de los archivos que contienen las palabras clave.
\item Calcular el nivel de importancia de cada archivo (PageRank).
\item Listar los archivos que contienen la palabra buscada, ordenados por importancia.
\end{itemize}

Para implementar estas funciones, se utilizaron estructuras de datos como grafos, listas enlazadas y tablas hash: Los grafos representan las relaciones entre documentos, las listas enlazadas permiten una navegación sencilla entre elementos relacionados, las tablas hash son esenciales para la búsqueda eficiente de palabras clave.

Este informe describe el proceso de desarrollo e implementación de Fly, detallando las decisiones de diseño y las técnicas utilizadas para crear un sistema eficiente de recuperación de información.

\subsection{Índice invertido}
El índice invertido es una estructura de datos que almacena los archivos que contienen un contenido específico, en lugar de listar los contenidos de un archivo. Esto es útil para trabajar con grandes volúmenes de información, ya que facilita la búsqueda y la relación entre los archivos y su contenido.

\subsection{PageRank}
PageRank es un algoritmo que calcula la importancia de un archivo dentro de un conjunto de archivos. La idea principal es que los archivos más importantes tendrán más enlaces desde otros archivos. La fórmula para calcular la importancia es:

\begin{equation}
PR(A)=(1-d)+d\sum_{i=1}^{N}\frac{PR(L_i)}{C(L_i)}
\end{equation}
Donde:
\begin{itemize}
\item $A:$ Archivo para el cual se calcula el PageRank
\item $d:$ Factor de amortiguación (0.85)
\item $L_i:$ Páginas que enlazan a $A$
\item $C(L_i):$ Número de enlaces en $L_i$
\end{itemize}

\subsection{Algoritmos}



\subsection{Otros ejecutables}