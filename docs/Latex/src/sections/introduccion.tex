\section{Introducción}
Un motor de búsqueda es una herramienta que permite buscar y recuperar datos en una colección de archivos.

Particularmente, \textit{Fly} es un motor de búsqueda que indexa documentos enlazados entre sí mediante \textit{WikiLinks}\footnote{Un WikiLink es un tipo de enlace entre documentos de texto usado en la aplicación \href{https://obsidian.md/}{Obsidian.md} consistente en el siguiente formato: \texttt{[[nombre del archivo enlazado|Alias para el archivo de ser necesario]]}. Este formato tiene diversas facilidades para un sistema de archivos con las características deseadas razón por la cuál fue utilizado en este proyecto.}. De estos archivos este motor extrae todas las palabras \textit{relevantes} permitiendo al usuario buscar una palabra y encontrar las coincidencias de la misma a lo largo del sistema de archivos. Para llevar a cabo esta tarea, Fly utiliza técnicas como el \textit{índice invertido} (para la indexación de palabras) y el \textit{PageRank} (para la jerarquización de los archivos analizados).

Las principales funciones de \textbf{Fly} son:
\begin{enumerate}
    \item Generar un grafo con los archivos de un directorio, almacenando en cada nodo un archivo y relacionándolo con los archivos que lo enlazan y aquellos a los que se enlaza.
    \item Generar un índice invertido para almacenar la información de cada palabra relevante en el sistema de archivos.
    \item Calcular el nivel de importancia de cada uno de los archivos en el grafo, utilizando el algoritmo de PageRank.
    \item Realizar búsquedas eficientes sobre el índice invertido y el grafo permitiendo al usuario encontrar coincidencias entre palabras y sus ubicaciones en cada archivo donde aparecen.
\end{enumerate}

Todas estas funciones se implementaron mediante el uso de estructuras de datos tales como grafos, listas enlazadas y tablas hash.

El presente informe describe el proceso de desarrollo e implementación de Fly, detallando las decisiones de diseño y las técnicas utilizadas para crear un sistema eficiente de busqueda y recuperación de información.