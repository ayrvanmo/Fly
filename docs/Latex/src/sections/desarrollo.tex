\section{Planteamiento del desarrollo del proyecto}
\rhostart{}

Para iniciar con el desarrollo del proyecto,  se establecieron cuatro características principales:

\begin{itemize}
\item \textbf{Index invertido:} Crear un índice invertido para almacenar los archivos que contienen una palabra específica, en lugar de listar los contenidos de un archivo. Esta característica resulta fundamental para el funcionamiento del programa, ya que es lo que permite identificar los archivos que contienen una palabra específica. Para implementar esto, surgieron dudas en cómo manejar los archivos y palabras, cómo almacenar que un archivo pertenece a una palabra, y cómo realizar la búsqueda de la forma más eficiente posible.
\item \textbf{PageRank:} Calcular el nivel de importancia de cada archivo (PageRank) utilizando el algoritmo de PageRank.
\item \textbf{Grafos:} Crear un grafo para representar las relaciones entre los archivos y sus contenidos.
\item \textbf{Manejo de archivos:} Investigar como crear una función de manejo de archivos para identificar los archivos regulares de texto plano en el directorio de entrada y en aquellos sub-directorios de éste y procesarlos de manera eficiente, evitando la lectura de archivos que no sean de este tipo y continuar con el archivo siguiente.
\end{itemize}

Con esto en mente, entonces se planteó el desarrollo del proyecto, dividiéndolo en cuatro etapas: Creación de la estructura de datos, creación de las funciones y algoritmos, el ensamblaje del programa y la optimización del código.