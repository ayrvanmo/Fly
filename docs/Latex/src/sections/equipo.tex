\section{Gestión del equipo de trabajo}
\rhostart{}

El equipo consto de 4 personas, donde entre éstas se tomaron decisiones y se distribuyeron los puntos importante del motor de busqueda para realizar esta tarea. Para mejorar el orden y la eficiencia implementamos \textbf{normas de codificación}, establecer fechas para tareas pendientes, y por consecuencia reuniones para acordar las siguientes tareas.

\subsection{Normas de codificación}
Con respecto al nombramiento de las constantes, estas fueron llamadas con el formato \textbf{SCREAMING\_SNAKE\_CASE}, y las variables se llamaron respectivamente con el formato \textbf{camelCase}. También a las funciones se acordó utilizar el formato \textbf{snake\_case}, y colocar las llaves de apertura en la siguiente linea, pero por otro lado aquellas llaves de apertura de otros bloques de codigo se colocarán justo al lado de su linea final, evitando omitir llaves en caso de existir únicamente una sentencia.

ejemplo de referencia: \textbf{https://github.com/ayrvanmo/ForKing/blob/master/docs/forKingDoc.pdf}

\subsection{Lista de tareas y organización}
Teniendo ya realizada las normas de codificación, se establecieron objetivos principales (Puntos importantes del programa), y objetivos secundarios o a corto plazo, los cuales fueron señalados para mantener un control sosbre el flujo de trabajo y repartirlos equitativamente pensando en los puntos fuertes de cada integrante del equipo; Por ejemplo los objetivos principales que designamos fueron asignados a cada integrante la codificación de cada una de ellas.

\subsection{Reuniones semanales}
Durante cada semana, se hacia por lo menos un día de reunión para conversar y gestionar los trabajos listos, pendientes y mejoras sobre estos mismos, y también para trabajar en aquellos puntos más complicados del programa. También se acordo que durante cada semana puedan realizarse reuniones casuales para abordar problemas sobre codificación simples y solucionarlos en conjunto. Las reuniones ya designadas previamente serian de manera presencial y en caso de que un integrante no pueda asistir éste debe ver la posibilidad de conectarse mediante llamada o videollamada para mantenerse actualizado sobre los objetivos y lo conversado en aquella reunión. En el caso de las reuniones casuales, estás pueden ser realizadas entre solo dos integrantes con la previa disponibilidad de ambos.