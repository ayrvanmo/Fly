\section{Gestión del equipo de trabajo}
El equipo de desarrollo de este proyecto fué conformado por cuatro integrantes; este proyecto por el tiempo de implementación requería de una excelente coordinación para lograr un buen resultado.

Para mejorar el orden y la eficiencia se implementaron \textit{normas de codificación}, así como el establecimiento de fechas para tareas pendientes, y por consecuencia reuniones para acordar los siguientes objetivos.

\subsection{Normas de codificación}
Con respecto al nombramiento de las constantes, estas fueron llamadas con el formato \texttt{SCREAMING\_SNAKE\_CASE}, y las variables se llamaron respectivamente con el formato \texttt{camelCase}. También a las funciones se acordó utilizar el formato \texttt{snake\_case}, y colocar las llaves de apertura en la siguiente linea, por otro lado aquellas llaves de apertura de otros bloques de codigo se colocarán justo al lado de su linea final, evitando omitir llaves en caso de existir únicamente una sentencia.

ejemplo de referencia (anterior proyecto bajo las mismas normas de codificación): \href{https://github.com/ayrvanmo/ForKing/blob/master/docs/forKingDoc.pdf}{Informe acerca del proyecto ForKing}.

\subsection{Lista de tareas y organización}
Teniendo ya realizada las normas de codificación, se establecieron objetivos principales (Puntos importantes del programa) a largo plazo, y objetivos secundarios para resolver a corto plazo.

Estos fueron escritos para  mantener un control sobre el flujo de trabajo y repartirlos equitativamente pensando en los puntos fuertes de cada integrante del equipo.